% Sets up custom commands and flags for school-specific behaviors
% Also assigns default values, but these can be overwritten with school-specific
% values in the corresponding header file.
% Some default values are placeholders (marked in red) and these MUST be 
% overwritten in the school-specific headers!
\newcommand{\schoolname}{\textcolor{red}{COLLEGE\_NAME}\xspace} 
\newcommand{\schoolnamelong}{\textcolor{red}{COLLEGE\_NAME\_LONG}\xspace}
\newcommand{\schooladdress}{\textcolor{red}{47 Generic St., City, ST 99999}\xspace}
\newcommand{\schoolintrocourses}{\textcolor{red}{[school-specific intro course numbers]}\xspace}
\newcommand{\schooladvcourses}{\textcolor{red}{[school-specific AI/ML/NLP course numbers]}\xspace}
\newcommand{\chairname}{\textcolor{red}{Prof. Search Chair}\xspace}
\newcommand{\chairlastname}{\textcolor{red}{Prof. Chair}\xspace}
\newcommand{\chairtitle}{\textcolor{red}{Chair, Faculty Search Committee}\xspace}
\newcommand{\department}{Department of Computer Science\xspace}
\newcommand{\position}{\textcolor{red}{POSITION\_NAME (POSITION\_ID)}\xspace}

\newif\iflongdei
\longdeitrue 
\newif\ifappendix
\appendixtrue 
\newif\ifliberalarts
\liberalartstrue
\newif\ifsupplementals
\supplementalsfalse

\newcommand{\materials}{my latest CV, a graduate transcript, and statements on teaching, research, and commitment to diversity and inclusion.\xspace}

\newcommand{\coverteachingpara}{%
\textcolor{red}{[school-specific segment on teaching philosophy]}
}

\newcommand{\coverresearchpara}{%
\textcolor{red}{[school-specific segment on research]}
}

\newcommand{\lateachingintro}{%
Liberal Arts education played a formative role in my approach to teaching---my love of teaching was first sparked during my undergraduate studies at Harvey Mudd College.
That was where I got my first (small) taste of what it is like to be an instructor, through my participation in the CS department's ``grutoring'' program---similar to TA positions found at larger universities, but with more emphasis on direct interaction with students.
This early experience of engaging directly with individual students, and seeing the specific things they struggled with, strongly influenced how I think about teaching.
To this day, I approach teaching by starting from a student's perspective: for any given course, what is a student most likely to struggle with, and what's the most effective way to help them understand it?
While the specific answers to these questions will vary with each course and each student cohort, there are nevertheless a few concrete strategies which I have found to be good \emph{starting points} for any course: \textbf{teaching with narratives}, \textbf{leveraging technology for interactivity}, and \textbf{building inclusive learning environments}.%
}

\newcommand{\genteachingintro}{%
\textcolor{red}{TODO WRITE ME}
}

\newcommand{\lanarrativeend}{%
This makes it particularly valuable for a Liberal Arts setting, which emphasizes drawing connections between all parts of a student's education.
In such a setting, one way I could imagine explicitly drawing out such connections in an intro-level course would be to build narratives based on content from other core courses.
For example, using real problems faced in Digital Humanities as motivating examples for lectures would help connect the lecture content to students' core humanities coursework.
This approach has the potential to hit ``two birds with one stone'', advancing the College's mission while simultaneously grounding the core CS concepts to make them more approachable.%
}

\newcommand{\gennarrativeend}{%
\textcolor{red}{TODO WRITE ME}
}

\newcommand{\lainteractionend}{This solution might work particularly well in a small classroom setting like that found at \schoolname.}

\newcommand{\geninteractionend}{%
\textcolor{red}{TODO WRITE ME}
}

\newcommand{\laresearchintro}{%
This research agenda is particularly well suited to a liberal arts institution like \schoolname, giving undergraduate students an opportunity to work with cutting-edge technology and simultaneously study the impact of this technology on society.
}

\newcommand{\genresearchintro}{%
\textcolor{red}{TODO WRITE ME}
}

\newcommand{\laresearchclosing}{%
Overall, I believe that my research agenda's unique mix of research and engineering components makes it a perfect fit for a primarily-undergraduate institution like \schoolname, and I am excited for the opportunity to continue working with talented undergraduates to pursue this groundbreaking and socially impactful research.
}

\newcommand{\genresearchclosing}{%
Overall, I believe that my research agenda's unique mix of research and engineering components makes it ideally suited as a learning experience for undergraduate students regardless of their post-graduation goals, and I am excited for the opportunity to continue working with talented undergraduates to pursue this groundbreaking and socially impactful research.
}
