% Sets up custom commands and flags for school-specific behaviors
% Also assigns default values, but these can be overwritten with school-specific
% values in the corresponding header file.
% Some default values are placeholders (marked in red) and these MUST be 
% overwritten in the school-specific headers!
\newcommand{\schoolname}{\textcolor{red}{COLLEGE\_NAME}\xspace} 
\newcommand{\schoolnamelong}{\textcolor{red}{COLLEGE\_NAME\_LONG}\xspace}
\newcommand{\schooladdress}{\textcolor{red}{47 Generic St., City, ST 99999}\xspace}
\newcommand{\schoolintrocourses}{\textcolor{red}{[school-specific intro course numbers]}\xspace}
\newcommand{\schooladvcourses}{\textcolor{red}{[school-specific AI/ML/NLP course numbers]}\xspace}
\newcommand{\chairname}{\textcolor{red}{Prof. Search Chair}\xspace}
\newcommand{\chairlastname}{\textcolor{red}{Prof. Chair}\xspace}
\newcommand{\chairtitle}{\textcolor{red}{Chair, Faculty Search Committee}\xspace}
\newcommand{\department}{Department of Computer Science\xspace}
\newcommand{\position}{\textcolor{red}{POSITION\_NAME (POSITION\_ID)}\xspace}

\newif\iflongdei
\longdeitrue 
\newif\ifappendix
\appendixtrue 
\newif\ifliberalarts
\liberalartstrue
\newif\ifsupplementals
\supplementalsfalse

\newcommand{\materials}{my latest CV, a graduate transcript, and statements on teaching, research, and commitment to diversity and inclusion.\xspace}

\newcommand{\coverteachingpara}{%
\textcolor{red}{[school-specific segment on teaching philosophy]}
}

\newcommand{\coverresearchpara}{%
\textcolor{red}{[school-specific segment on research]}
}

\newcommand{\lateachingintro}{%
Liberal Arts education played a formative role in my approach to teaching: my love of teaching was first sparked during my undergraduate studies at Harvey Mudd College.
That was where I got my first (small) taste of what it is like to be an instructor, through my participation in the CS department's ``grutoring'' program---similar to TA positions found at larger universities, but with more emphasis on direct interaction with students.
This early experience of engaging directly with individual students, and seeing the specific things they struggled with, strongly influenced how I think about teaching.
To this day, I approach teaching by starting from a student's perspective: for any given course, what is a student most likely to struggle with, and what's the most effective way to help them understand it?
While the specific answers to these questions will vary with each course and each student cohort, there are nevertheless a few concrete strategies which I have found to be good \emph{starting points} for any course: \textbf{teaching with narratives}, \textbf{leveraging technology for interactivity}, and \textbf{building inclusive learning environments}.%
}

\newcommand{\genteachingintro}{%
My love of teaching was first sparked during my undergraduate studies at Harvey Mudd College.
That was where I got my first (small) taste of what it is like to be an instructor, through my participation in the CS department's ``grutoring'' program---similar to TA positions found at larger universities, but with more emphasis on direct interaction with students.
This early experience of engaging directly with individual students, and seeing the specific things they struggled with, strongly influenced how I think about teaching.
Even as I have moved on to teaching in larger classroom settings during my graduate career, I continue to approach teaching by starting from a student's perspective: for any given course, what is a student most likely to struggle with, and what's the most effective way to help them understand it?
While the specific answers to these questions will vary with each course and each student cohort, there are nevertheless a few concrete strategies which I have found to be good \emph{starting points} for any course: \textbf{teaching with narratives}, \textbf{leveraging technology for interactivity}, and \textbf{building inclusive learning environments}.%
}

\newcommand{\lanarrativeend}{%
This makes it particularly valuable for a Liberal Arts setting, which emphasizes drawing connections between all parts of a student's education.
In such a setting, I would envision building narratives based on real-world applications of CS in many different fields---encompassing not only physical and social sciences, but also arts and humanities (through subfields like Digital Humanities and Library Science).
Doing so will help my CS teaching feel more directly relevant and approachable to students of varied interests, CS majors and non-majors alike.%
}

\newcommand{\gennarrativeend}{%
In addition to helping make the core concepts of CS feel more grounded and approachable, I also believe that teaching with narratives has the additional benefit of introducing students to the many real-world applications of Computer Science, which will benefit both CS majors and non-majors alike.
}

\newcommand{\lainteractionend}{I would also encourage students to help each other by working together in live coding sessions---an approach particularly well-suited to a small classroom setting like that found at \schoolname.}

\newcommand{\geninteractionend}{I would further complement this by encouraging students to work together in small groups for live coding sessions, which can help this technique scale to very large classes.}

\newcommand{\lateachingend}{%
Finally, I also look forward to the opportunity to expand \schoolname's course offerings; in particular, I would be interested in developing a new course on Computational Social Science, a topic that fits well with the interdisciplinary and well-rounded spirit of a Liberal Arts education.
}

\newcommand{\genteachingend}{%
Finally, I also look forward to the opportunity to expand \schoolname's course offerings; in particular, I would be interested in developing a new course on Computational Social Science, an increasingly important field of study that is relevant to work in social media and human-centered computing.
}

\newcommand{\laresearchintro}{%
This research agenda is particularly well suited to a liberal arts institution like \schoolname, giving undergraduate students an opportunity to work with cutting-edge technology and simultaneously study the impact of this technology on society.
}

\newcommand{\genresearchintro}{%
With its focus on real-world applications, this research agenda is particularly well suited for undergraduate involvement, offering undergraduates an opportunity to apply the programming skills they have learned in class to a continuously evolving research setting.
}

\newcommand{\laresearchclosing}{%
Overall, I believe that my research agenda's unique mix of research and engineering components makes it a perfect fit for a primarily-undergraduate institution like \schoolname, and I am excited to continue working with talented undergraduates to pursue this groundbreaking and socially impactful research.
}

\newcommand{\genresearchclosing}{%
Overall, I believe that my research agenda's unique mix of research and engineering components makes it ideally suited as a learning experience for undergraduates, and I am excited to continue working with talented undergraduates to pursue this groundbreaking and socially impactful research.
}

\newcommand{\deioutreach}{%
For instance, inspired by my experience as a Peer Academic Liaison, I would be interested in pioneering a similar program at \schoolname, either at a whole-college level or as a department level program that could offer more targeted support and advice.%
}
