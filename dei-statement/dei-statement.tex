\documentclass[12pt,letterpaper]{article}
\usepackage[margin=0.75in]{geometry}
\usepackage{graphicx}
%\usepackage[bf,tiny,compact]{titlesec}
\usepackage{times}
\usepackage{xcolor}
\usepackage{xspace}
\usepackage{enumerate}
\usepackage{enumitem}
\usepackage{titlesec}
\usepackage{etoolbox}
\usepackage{hyperref}


\setlist{nosep}

\makeatletter
\def\@maketitle{%
    {\centering\fontsize{12pt}{14pt}\selectfont\textbf{\@title}\par}
}
\patchcmd{\@footnotetext}{\footnotesize}{\fontsize{10pt}{12pt}\selectfont}{}{}
\makeatother

\titleformat{\section}[runin]{\normalfont\normalsize\bfseries}{}{0em}{}[.]
\titlespacing*{\section}{0em}{0.25\baselineskip}{0.5em}

%% LOGIC FOR CUSTOMIZING THE STATEMENT FOR INDIVIDUAL SCHOOLS
% set up common defines (commands and boolean flags)
% Sets up custom commands and flags for school-specific behaviors
% Also assigns default values, but these can be overwritten with school-specific
% values in the corresponding header file.
% Some default values are placeholders (marked in red) and these MUST be 
% overwritten in the school-specific headers!
\newcommand{\schoolname}{\textcolor{red}{COLLEGE\_NAME}\xspace} 
\newcommand{\schoolnamelong}{\textcolor{red}{COLLEGE\_NAME\_LONG}\xspace}
\newcommand{\schooladdress}{\textcolor{red}{47 Generic St., City, ST 99999}\xspace}
\newcommand{\schoolintrocourses}{\textcolor{red}{[school-specific intro course numbers]}\xspace}
\newcommand{\schooladvcourses}{\textcolor{red}{[school-specific AI/ML/NLP course numbers]}\xspace}
\newcommand{\chairname}{\textcolor{red}{Prof. Search Chair}\xspace}
\newcommand{\chairlastname}{\textcolor{red}{Prof. Chair}\xspace}
\newcommand{\chairtitle}{\textcolor{red}{Chair, Faculty Search Committee}\xspace}
\newcommand{\department}{Department of Computer Science\xspace}
\newcommand{\position}{\textcolor{red}{POSITION\_NAME (POSITION\_ID)}\xspace}

\newif\iflongdei
\longdeitrue 
\newif\ifappendix
\appendixtrue 
\newif\ifliberalarts
\liberalartstrue

\newcommand{\materials}{my latest CV, a graduate transcript, and statements on teaching, research, and commitment to diversity and inclusion.\xspace}

\newcommand{\coverteachingpara}{%
\textcolor{red}{[school-specific segment on teaching philosophy]}
}

\newcommand{\coverresearchpara}{%
\textcolor{red}{[school-specific segment on research]}
}

\newcommand{\lateachingintro}{%
Liberal Arts education played a formative role in my approach to teaching---my love of teaching was first sparked during my undergraduate studies at Harvey Mudd College.
That was where I got my first (small) taste of what it is like to be an instructor, through my participation in the CS department's ``grutoring'' program---similar to TA positions found at larger universities, but with more emphasis on direct interaction with students.
This early experience of engaging directly with individual students, and seeing the specific things they struggled with, strongly influenced how I think about teaching.
To this day, I approach teaching by starting from a student's perspective: for any given course, what is a student most likely to struggle with, and what's the most effective way to help them understand it?
While the specific answers to these questions will naturally vary with each course and each student cohort, there are nevertheless a few concrete strategies which, broadly speaking, I have found to be good \emph{starting points} for any course: \textbf{teaching with narratives}, \textbf{leveraging technology for interactivity}, and \textbf{building inclusive learning environments}.%
}

\newcommand{\genteachingintro}{%
\textcolor{red}{TODO WRITE ME}
}

\newcommand{\lanarrativeend}{%
This makes it particularly valuable for a Liberal Arts setting, which emphasizes drawing connections between all parts of a student's education.
In such a setting, one way I could imagine explicitly drawing out such connections in an intro-level course would be to build narratives based on content from other core courses.
For example, using real problems faced in Digital Humanities as motivating examples for lectures would help connect the lecture content to students' core humanities coursework.
This approach has the potential to hit ``two birds with one stone'', advancing the College's mission while simultaneously grounding the core CS concepts to make them more approachable.%
}

\newcommand{\gennarrativeend}{%
\textcolor{red}{TODO WRITE ME}
}

\newcommand{\lainteractionend}{I believe this solution is particularly well suited to the small-class environment of a liberal arts college.}

\newcommand{\geninteractionend}{%
\textcolor{red}{TODO WRITE ME}
}

% import the school-specific header to make the magic happen!
% Not an actual school; meant to provide placeholder values
% for school-specific variables for the purposes of compiling
% generic versions of the statements
\renewcommand{\schoolname}{\textcolor{red}{COLLEGE\_NAME}\xspace}
\renewcommand{\schoolintrocourses}{\textcolor{red}{[school-specific intro course numbers]}\xspace}
\renewcommand{\schooladvcourses}{\textcolor{red}{[school-specific AI/ML/NLP course numbers]}\xspace}
\longdeitrue
\appendixtrue
\liberalartstrue
%% END CUSTOMIZATION LOGIC

\title{Statement on Diversity, Equity, and Inclusion}

\begin{document}
\maketitle

{\centering Jonathan P. Chang \par}

\vspace{0.5\baselineskip}

My approach to teaching, as I have laid out in my teaching statement, is to start by thinking about the course and the material from a student's perspective.
It is important to me that when I do this, I truly account for \emph{all} of my students---which inherently requires that I consider perspectives that may be wildly different from my own (cis-gendered, male, middle-class, able-bodied) perspective.
Consequently, a central tenet of my teaching is addressing the challenges in diversity, equity, and inclusion that CS as a field continues to face.
I think about this not only through the lens of building a more inclusive classroom, but also at a higher level: promoting equity and inclusion in the community I belong to (whether that be my undergraduate institution Harvey Mudd, my graduate institution Cornell, or, looking ahead, \schoolname), and training students who will broadly advance this cause in their future careers.

\section{Diversity, Equity, and Inclusion in the Classroom}
At the most direct level, accounting for diverse student backgrounds in my teaching implies taking steps to design a course environment that is accessible and welcoming to all.
In pursuit of this goal, there are a few concrete strategies I have tried out in my teaching career thus far, both as co-instructor of record for CS4300 ``Information and Language'' and as a designer and instructor of workshops for the Cornell Center for Social Sciences (CCSS).

In my teaching statement, I detailed a number of steps I took in my CS4300 teaching to make existing course structures more inclusive.
One of these was introducing alternative modes of in-class participation, such as online polls and forum-style discussions, so that students who are uncomfortable with speaking in front of the whole class, or unable to do so, nevertheless have ways to contribute their thoughts.
Another step was overseeing the development of a brand new project server and template for the course final project, which abstracted away a lot of the boilerplate parts of app development.
This system helped ensure that students who have not had prior opportunities to practice app development skills were on a more level playing field with those who had such opportunities (e.g., through internships), and made evaluation more fair by helping the course staff focus on the part of the project we actually cared about evaluating, namely the information retrieval aspect.
I believe these efforts were part of why students ultimately rated the course an average 4.44 out of 5 on diversity and inclusion in their end-of-semester evaluations.

More recently, I have come to focus on another diversity-related goal: ensuring that students both know how to ask for help and are comfortable doing so.
I firmly believe knowing how to ask for help---that is, knowing what resources are best for a given problem and how to best word your question---is a learned skill, just like knowing how to code.
But in my conversations with students I have found that they sometimes instead view asking for help as a weakness, an effect that may be exacerbated among underrepresented groups due to impostor syndrome.
Accordingly, I have started incorporating segments about asking for help into my CCSS workshops, especially those targeted at intro-level programming audiences.
These segments involve not only explaining that I have to seek help myself all the time in my work, but also demonstrating this process as part of the workshop---for instance by showing how to find and interpret documentation online, going over effective strategies for doing targeted web searches about coding errors, and directly asking my co-instructors to offer their ideas.

\section{Diversity, Equity, and Inclusion in the Community}
The work needed to build inclusive learning environments does not stop at the level of individual course design: students' sense of belonging is affected not only by the design of lectures and assignments, but also by whether they feel welcomed by the broader community.
Therefore, another major part of my efforts in diversity, equity, and inclusion is ensuring that the communities I am a part of are welcoming to students of all backgrounds.

This aspect of my work has extended through both my undergraduate and graduate careers.
During my undergraduate studies at Harvey Mudd College, I served for two years as a Peer Academic Liaison, a then-new program designed to ensure that first-year students would have a designated upper-year peer to be a point of contact for academic questions or concerns.
My conversations with students as part of this role challenged me to think carefully about how best to help students whose lived experience, and subsequent needs, might be very different from my own.
I went on to follow up on this thread in my Ph.D. studies by volunteering as a mentor in the Cornell CS Student-Applicant Support Program.
This ongoing program helps prospective students from underrepresented groups by connecting them to a Ph.D. mentor who can help them decide what they want out of graduate school and advise them on how to approach the application process.

The running theme through both of these experiences is \emph{mentorship}.
Looking back on my education, I recognize that a major part of my success has been that my background afforded me a lot of opportunities to receive support along my educational journey, be that from family, teachers, tutors, or peers.
Not all students have such opportunities, especially when it comes to underrepresented groups.
I therefore hope that my participations in programs like the above can help bridge this opportunity gap for at least some students.
While I recognize that such work alone can never fully address the wider systemic issues that are faced in both education and the tech sector, I nevertheless believe that they can still contribute to building diverse communities that are more aware of and sensitive to those issues.

\section{Fostering Student Commitment to Diversity, Equity and Inclusion}
Computer Science continues to reshape our world, and I know that the students I teach can go on to have a great impact on society through the work they do.
Therefore, my responsibility as a socially responsible instructor is to ensure that this impact is positive.
To this end, throughout my teaching I aim to make students aware of the broader impacts of the skills they are learning, and the ways in which technology can both help and hinder the cause of justice and equity. 
I do this both explicitly and implicitly.

At an explicit level, my lectures incorporate discussions of social implications of the material that was taught in the lecture.
For example, my CS4300 lecture on unsupervised machine learning techniques for natural language included a discussion of the social biases that these techniques can pick up and propagate.
Importantly, this meant more than simply asserting ``language models are biased''; rather, I connected this finding back to the fundamentals that the lecture had covered, in order to communicate at a technical level \emph{why} this bias occurs.
I then concluded this discussion by connecting it to best practices for data use, explaining to students that thinking about biases is a key step that they should always take when working with a new dataset, alongside other previously discussed steps like visualization and cleaning.

At an implicit level, I ensure that the values of diversity, equity, and inclusion are reflected in the examples I use in my teaching.
For instance, at a recent CCSS workshop covering how to do statistical tests in Python, the example I used for running correlational analyses was a public dataset of student economic data and standardized test scores.
Students going through the exercise would therefore end up reproducing the social science research showing that standardized test scores are strongly correlated with economic variables such as income bracket.
This strategy can also connect back to my more explicit techniques, such as by holding a follow up discussion for students to reflect on the findings and think about what they would do if they came across such findings during their own data explorations.

\end{document}