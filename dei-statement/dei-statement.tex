\documentclass[12pt,letterpaper]{article}
\usepackage[margin=0.75in]{geometry}
\usepackage{graphicx}
%\usepackage[bf,tiny,compact]{titlesec}
\usepackage{times}
\usepackage{xcolor}
\usepackage{xspace}
\usepackage{enumerate}
\usepackage{enumitem}
\usepackage{titlesec}
\usepackage{etoolbox}
\usepackage{hyperref}


\setlist{nosep}

\makeatletter
\def\@maketitle{%
    {\centering\fontsize{12pt}{14pt}\selectfont\textbf{\@title}\par}
}
\patchcmd{\@footnotetext}{\footnotesize}{\fontsize{10pt}{12pt}\selectfont}{}{}
\makeatother

\titleformat{\section}{\normalfont\large\bfseries\scshape}{}{0em}{}
\titlespacing*{\section}{0em}{0.25\baselineskip}{0em}
\titleformat{\subsection}[runin]{\normalfont\normalsize\bfseries}{}{0em}{}[.]
\titlespacing*{\subsection}{0em}{0.25\baselineskip}{0.5em}

%% LOGIC FOR CUSTOMIZING THE STATEMENT FOR INDIVIDUAL SCHOOLS
% set up common defines (commands and boolean flags)
% Sets up custom commands and flags for school-specific behaviors
% Also assigns default values, but these can be overwritten with school-specific
% values in the corresponding header file.
% Some default values are placeholders (marked in red) and these MUST be 
% overwritten in the school-specific headers!
\newcommand{\schoolname}{\textcolor{red}{COLLEGE\_NAME}\xspace} 
\newcommand{\schoolnamelong}{\textcolor{red}{COLLEGE\_NAME\_LONG}\xspace}
\newcommand{\schooladdress}{\textcolor{red}{47 Generic St., City, ST 99999}\xspace}
\newcommand{\schoolintrocourses}{\textcolor{red}{[school-specific intro course numbers]}\xspace}
\newcommand{\schooladvcourses}{\textcolor{red}{[school-specific AI/ML/NLP course numbers]}\xspace}
\newcommand{\chairname}{\textcolor{red}{Prof. Search Chair}\xspace}
\newcommand{\chairlastname}{\textcolor{red}{Prof. Chair}\xspace}
\newcommand{\chairtitle}{\textcolor{red}{Chair, Faculty Search Committee}\xspace}
\newcommand{\department}{Department of Computer Science\xspace}
\newcommand{\position}{\textcolor{red}{POSITION\_NAME (POSITION\_ID)}\xspace}

\newif\iflongdei
\longdeitrue 
\newif\ifappendix
\appendixtrue 
\newif\ifliberalarts
\liberalartstrue

\newcommand{\materials}{my latest CV, a graduate transcript, and statements on teaching, research, and commitment to diversity and inclusion.\xspace}

\newcommand{\coverteachingpara}{%
\textcolor{red}{[school-specific segment on teaching philosophy]}
}

\newcommand{\coverresearchpara}{%
\textcolor{red}{[school-specific segment on research]}
}

\newcommand{\lateachingintro}{%
Liberal Arts education played a formative role in my approach to teaching---my love of teaching was first sparked during my undergraduate studies at Harvey Mudd College.
That was where I got my first (small) taste of what it is like to be an instructor, through my participation in the CS department's ``grutoring'' program---similar to TA positions found at larger universities, but with more emphasis on direct interaction with students.
This early experience of engaging directly with individual students, and seeing the specific things they struggled with, strongly influenced how I think about teaching.
To this day, I approach teaching by starting from a student's perspective: for any given course, what is a student most likely to struggle with, and what's the most effective way to help them understand it?
While the specific answers to these questions will naturally vary with each course and each student cohort, there are nevertheless a few concrete strategies which, broadly speaking, I have found to be good \emph{starting points} for any course: \textbf{teaching with narratives}, \textbf{leveraging technology for interactivity}, and \textbf{building inclusive learning environments}.%
}

\newcommand{\genteachingintro}{%
\textcolor{red}{TODO WRITE ME}
}

\newcommand{\lanarrativeend}{%
This makes it particularly valuable for a Liberal Arts setting, which emphasizes drawing connections between all parts of a student's education.
In such a setting, one way I could imagine explicitly drawing out such connections in an intro-level course would be to build narratives based on content from other core courses.
For example, using real problems faced in Digital Humanities as motivating examples for lectures would help connect the lecture content to students' core humanities coursework.
This approach has the potential to hit ``two birds with one stone'', advancing the College's mission while simultaneously grounding the core CS concepts to make them more approachable.%
}

\newcommand{\gennarrativeend}{%
\textcolor{red}{TODO WRITE ME}
}

\newcommand{\lainteractionend}{I believe this solution is particularly well suited to the small-class environment of a liberal arts college.}

\newcommand{\geninteractionend}{%
\textcolor{red}{TODO WRITE ME}
}

% import the school-specific header to make the magic happen!
% Not an actual school; meant to provide placeholder values
% for school-specific variables for the purposes of compiling
% generic versions of the statements
\renewcommand{\schoolname}{\textcolor{red}{COLLEGE\_NAME}\xspace}
\renewcommand{\schoolintrocourses}{\textcolor{red}{[school-specific intro course numbers]}\xspace}
\renewcommand{\schooladvcourses}{\textcolor{red}{[school-specific AI/ML/NLP course numbers]}\xspace}
\longdeitrue
\appendixtrue
\liberalartstrue
%% END CUSTOMIZATION LOGIC

\title{Statement on Diversity, Equity, and Inclusion}

\begin{document}
\maketitle

{\centering Jonathan P. Chang \par}

\vspace{0.5\baselineskip}

My approach to teaching, as I have laid out in my teaching statement, is to start by thinking about the course and the material from a student's perspective.
It is important to me that when I do this, I truly account for \emph{all} of my students---which inherently requires that I consider perspectives that may be wildly different from my own (cis-gendered, male, middle-class, able-bodied) perspective.
For women, racial and ethnic minorities, the economically disadvantaged, and other underrepresented groups in CS, \emph{systemic barriers} continue to stand in the way of both entry to and success in our field.
Such barriers---including lack of role models and mentors, inability to financially afford tools and resources, and the unconscious biases and microaggressions that drive impostor syndrome and stereotype threat---can affect a student's life trajectory long before they even set foot on campus.
I therefore work to create learning environments that account for these existing inequities and empower all students to succeed, which requires efforts at multiple levels: the \textbf{classroom level} where I work to design more inclusive courses, the \textbf{community level} where I volunteer in initiatives to improve campus diversity and inclusion, and the \textbf{global level} where I encourage students to pursue equitable applications of CS through their future work and careers.

\section{Inclusive Course Design}
My approach for designing courses to be more inclusive and equitable starts from interrogating the unconscious assumptions underlying existing teaching paradigms, and how these assumptions may systematically exclude certain groups.
This can then inform the changes that need to be made to foster a more inclusive environment.
In my teaching statement, I describe the steps I took as co-instructor of record for CS4300 at Cornell to implement this strategy, which I elaborate on below.

\subsection{Past Efforts}
As one example, consider in-class participation: while it is quite common to see participation measured in terms of asking or answering questions in class, this approach fails to account for students who---for reasons ranging from cultural differences to disability---are either uncomfortable with speaking in front of the whole class or unable to do so at all.
To address this, I introduced a number of alternative avenues for participation in CS4300, such as in-class polls and quizzes and online forum-style discussion threads, so that such students have a chance to make their thoughts known.

Perhaps a more subtle example is course projects: while these are not inherently inequitable, more privileged students may have prior experience with creating big software projects (e.g., through coding camps or internships) that less privileged students do not.
Students with prior project experience may therefore get an unintentional advantage in that they can spend less time struggling with setup and boilerplate coding, and more time debugging and refining.
To help address this, I oversaw the development of a brand-new project server and template for the CS4300 final project, which abstracted away common boilerplate coding tasks and offered all students a common starting point to build upon.
Overall, CS4300 students seemed to recognize my efforts at creating a more inclusive learning environment, with end-of-semester evaluations rating the course an average 4.44 out of 5 on diversity and inclusion, alongside comments acknowledging how steps like the project server contributed to a more level playing field.\footnote{%
Selected quotes can be found in the teaching statement, with full quotes available
\ifsupplementals
in the supplemental materials.
\else
upon request.
\fi
}

\subsection{Future Plans}
In-class participation and course projects are, of course, only two out of many components of course design that can be reimagined to improve inclusion and equity.
Inspired in part by student feedback from CS4300, another component I would like to focus on in my future teaching is assignments and exams.
Building upon some strategies I have seen from my instructors and peers, I would like to explore ways to move away from the traditional paradigm of infrequent, topic-comprehensive exams, and towards a system of more frequent evaluations that are individually smaller and more topically focused.
This approach can advance inclusion not only by enabling greater flexibility (which accomodates students who may require the flexibility due to health or personal circumstance), but also by serving as a way to identify students in need of support early on, so I can make sure they get the help they need.

%More recently, I have come to focus on another diversity-related goal: ensuring that students both know how to ask for help and are comfortable doing so.
%I firmly believe knowing how to ask for help---that is, knowing what resources are best for a given problem and how to best word your question---is a learned skill, just like knowing how to code.
%But in my conversations with students I have found that they sometimes instead view asking for help as a weakness, an effect that may be exacerbated among underrepresented groups due to impostor syndrome.
%Accordingly, I have started incorporating segments about asking for help into my CCSS workshops, especially those targeted at audiences who are new to programming.
%These segments involve not only explaining that I have to seek help myself all the time in my work, but also demonstrating this process as part of the workshop---for instance by showing how to find and interpret documentation online, going over effective strategies for doing targeted web searches about coding errors, and directly asking my co-instructors to offer their ideas.

\section{Promoting Diversity, Equity, and Inclusion in my Community}
The work needed to build inclusive learning environments does not stop at the level of individual course design: students' sense of belonging also depends on feeling welcomed by the broader community.
Therefore, another major part of my efforts is volunteering in intiatives to promote diversity, equity, and inclusion within the communities where I work and study. 

\subsection{Past Efforts}
This aspect of my work has extended through both my undergraduate and graduate careers.
As an undergraduate at Harvey Mudd College, I served for two years as a Peer Academic Liaison, a then-new program designed to ensure that first-year students would have a designated upper-year peer to be a point of contact for academic questions or concerns.
Through the conversations I had with students as part of this role, I saw firsthand the ways in which their backgrounds and past opportunities continued to affect their trajectories in college, and to this day that experience informs my thinking about the different needs that students in my classroom may face.
I went on to follow up on this thread during my Ph.D. by volunteering as a mentor in the Cornell CS Student-Applicant Support Program.
This program, targeted towards underrepresented groups, connects prospective Ph.D. students to a current student mentor who can help them decide what they want out of graduate school and advise them on the application process.

\subsection{Future Plans}
The running theme through both of these experiences is \emph{mentorship}.
Lack of role models and mentors is one type of systemic barrier that faces underrepresented groups in CS, and while programs like the above are not a complete solution, they do go some way towards bridging this gap and providing students from these groups the support they deserve.
As faculty, I will be in a position to not only continue serving such mentorship roles, but also to direct students to mentoring and outreach opportunities, and even to create new opportunities.
\deioutreach

\section{Fostering Student Commitment to Equitable Applications of CS}
Computer Science continues to reshape our world, but there is no guarantee that this impact will always be positive.
As an educator, it is my responsibility to ensure that my students are aware of the ways CS can help or hinder the cause of diversity, equity, and inclusion, so that they are equipped to leave a positive impact on the world by working on equitable and responsible applications of CS.

\subsection{Past Efforts}
As an example of how I have promoted socially responsible use of CS in my teaching, my CS4300 lecture on unsupervised machine learning techniques for NLP included a discussion of how these techniques can reproduce and amplify existing racial and gender biases.
Importantly, this meant more than just showing examples of NLP models being biased; rather, I connected this finding back to the fundamentals that the lecture had covered, in order to communicate at a technical level \emph{why} this bias occurs.
I then concluded this discussion by asking students to think about potential sources of bias in datasets they were working with for their course project and other applications, framing such considerations as part of the best practices for data use (alongside technical steps like visualization and cleaning).

\subsection{Future Plans}
While I am most familiar with issues of bias in NLP since that is my research area, I am aware that similar issues are faced by other subfields such as computer vision and health technology.
In my future teaching, I would want to draw from examples across all these domains, which is especially important at the intro level since students in intro courses will come in with a variety of interests.
\ifliberalarts
In the interdisciplinary spirit of the Liberal Arts, I plan to consult with faculty in other departments to gather ideas for examples, and will also work to arrange guest lectures so students can hear firsthand perspectives on the societal implications of CS applications in various fields.
\else
In doing so, I will take full advantage of the resources of a world-class institution like \schoolname, reaching out to faculty in other departments to both gather ideas and arrange guest lectures so students can hear firsthand perspectives on the societal implications of CS applications in various fields.
\fi

%At an implicit level, I ensure that the values of diversity, equity, and inclusion are reflected in the examples I use in my teaching.
%For instance, at a recent CCSS workshop covering how to do statistical tests in Python, the example I used for running correlational analyses was reproducing an analysis in education research showing that standardized test scores are strongly correlated with economic variables such as income bracket.
%Examples like this are a way to show students firsthand how they can use the skills they are learning to reveal and understand social inequities.
%This strategy can also connect back to my more explicit techniques, such as by holding a follow-up discussion for students to reflect on the findings and think about what they would do if they came across similar findings in their own research.


%\section{Future Development}
%I recognize that the work of advancing diversity, equity, and inclusion, both in and out of the classroom, is never truly done, and I view my efforts thus far as first steps that I must now build upon.
%This means both refining the methods I have developed already, and coming up with new strategies and activities.
%In pursuit of this goal, I look forward to working with the diverse faculty and student body at \schoolname to share ideas.
%In my career thus far, a lot of my thinking has been shaped by interactions with colleagues, especially those from very different backgrounds, and I believe this will continue to be the case as I look ahead to my future career.

\end{document}