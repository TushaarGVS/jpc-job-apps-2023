\documentclass[11pt,letterpaper]{article}
\usepackage[margin=0.75in]{geometry}
\usepackage{graphicx}
%\usepackage[bf,tiny,compact]{titlesec}
\usepackage{times}
\usepackage{cite}
\usepackage{natbib}
\usepackage{titlesec}
\usepackage{etoolbox}
\usepackage{hyperref}


\makeatletter
\def\@maketitle{%
    {\centering\fontsize{12pt}{14pt}\selectfont\textbf{\@title}\par}
}
\patchcmd{\@footnotetext}{\footnotesize}{\fontsize{10pt}{12pt}\selectfont}{}{}
\makeatother


\renewcommand{\bibsection}{\noindent\textbf{References (works to which I contrubted have my name \underline{underlined})}\vspace{-6pt}}
\setlength{\bibsep}{0pt} % or use whatever dimension you want
\renewcommand{\bibfont}{\fontsize{10pt}{12pt}\selectfont} % or any other  appropriate font command


\renewcommand{\section}[1]{\vspace{0.25\baselineskip}\noindent\textbf{#1.}}
\renewcommand{\subsection}[1]{\vspace{0.25\baselineskip}\noindent\textit{#1.}}

%% LOGIC FOR CUSTOMIZING THE STATEMENT FOR INDIVIDUAL SCHOOLS
\newcommand{\schoolname}{COLLEGE\_NAME}
%% END CUSTOMIZATION LOGIC

\title{SOME FANCY TITLE HERE}

\begin{document}
\maketitle

\begin{center}
Teaching Statement --- Jonathan P. Chang
\end{center}

% TODO different opening paragraph for non liberal arts colleges
Liberal arts education played a formative role in my approach to teaching---my love of teaching was first sparked at Harvey Mudd College, where I did my undergraduate degree.
Specifically, Harvey Mudd was where I got my first small taste of what it is like to be an instructor, through my participation in the CS department's ``grutoring'' program---similar to the TA positions found at larger universities, but with a greater emphasis on direct interaction with students.
Though I initially joined the program just out of curiosity (and admittedly also for the chance to get paid!), I quickly fell in love with both the challenge of finding the best way to explain a difficult concept to a student, and the satisfaction of seeing the ``aha'' moment where they finally get it.
Since then, my later experiences as an actual instructor have been spent chasing those ``aha'' moments---and more concretely, developing and implementing strategies to make those moments happen more often and reliably.

Subsequently, I have developed a guiding philosophy for my teaching: I aim for learning environments where students can feel a sense of \emph{ownership} and \emph{accomplishment} over their learning.
Students in my classroom should not feel like mere receivers of knowledge, but instead should feel like active participants working together with me and each other to arrive at a shared understanding of the material.
During my PhD, I have had the chance to implement this philosophy in two disparate settings: a more traditional classroom setting as co-instructor of record for CS4300 ``Information and Language'',\footnote{\url{https://classes.cornell.edu/browse/roster/SP23/class/CS/4300}} and a more personal and ad-hoc setting as a designer and instructor of data science workshops for the Cornell Center for Social Sciences (CCSS).\footnote{\url{https://socialsciences.cornell.edu/computing-and-data/workshops-and-training}}
These experiences serve to illustrate three concrete strategies that emerge from my teaching philosophy: \textbf{organizing lessons around narrative framings}, \textbf{leveraging technology for interactivity}, and \textbf{developing course content with diverse student backgrounds in mind}.

\section{Organizing Lessons Around Narrative Framings}
In the summer of 2022, I was designing a brand-new Introduction to Python workshop that I would teach for CCSS.
Like most CCSS workshops, this 4-part bootcamp was designed for a target audience of social science researchers with little to no technical background, who may have never done any programming before.
Explaining core programming concepts---like variables, functions, and logic---to first-time programmers is known to be a challenging problem.
My approach for addressing this challenge was to ground those abstract concepts in a social science narrative: using a real dataset sourced from a social science researcher at CCSS, the workshop asked students to imagine that they were using this data for their own research, then framed the lessons in terms of steps a researcher would need to take in preparing and analyzing the data.
For example, the need to store and represent datapoints motivates the concept of variables, and the need to repeatedly perform common operations across all the data motivates the concepts of loops and functions.

My work on designing the Introduction to Python workshop serves as an illustrative example of one of my preferred teaching strategies: organizing lesson plans around a central narrative, with each lesson having a clearly defined purpose within that narrative.
This strategy arises directly from my core philosophy of having students feel ownership over their learning: I believe these narratives signal to students that the course is really designed \emph{for them}, in that they are gaining skills that directly connect to their own goals and needs as opposed to just rotely absorbing information with no rhyme or reason.
Feedback from the workshop participants validates this belief, as several students told us that they learned much more from our workshop than in their previous attempts to learn Python or other programming skills.
For instance, one student wrote in their feedback form ``I finally understood how to use Python; I have taken other classes at Cornell but none were so comprehensive as yours'', while another commented that ``This is one of the best tutorials I've been to. The instructors were knowledgeable, kind, and better at communicating/explaining code then most I've seen.''

Admittedly, coming up with a central organizing narrative for a 4-part workshop is quite a different prospect from coming up with one for a semester-long course, which is much broader in scope.
But it is still possible to find organizing narratives for individual topics within a course; for example, as an organizing narrative for the CS4300 lectures on unsupervised machine learning methods for information retrieval, I described an example final project for the course (as the students were working on their own final projects at the time), and motivated each unsupervised machine learning method by demonstrating what functionality it could add to the project.  
As I look forward to a future teaching career, I will continue to iterate on this strategy to develop novel ways of organizing and motivating content in courses at \schoolname.


\end{document}
