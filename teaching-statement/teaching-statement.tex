\documentclass[12pt,letterpaper]{article}
\usepackage[margin=0.75in]{geometry}
\usepackage{graphicx}
%\usepackage[bf,tiny,compact]{titlesec}
\usepackage{times}
\usepackage{xcolor}
\usepackage{xspace}
\usepackage{enumerate}
\usepackage{enumitem}
\usepackage{titlesec}
\usepackage{etoolbox}
\usepackage{hyperref}


\setlist{nosep}

\makeatletter
\def\@maketitle{%
    {\centering\fontsize{12pt}{14pt}\selectfont\textbf{\@title}\par}
}
\patchcmd{\@footnotetext}{\footnotesize}{\fontsize{10pt}{12pt}\selectfont}{}{}
\makeatother

\titleformat{\section}[runin]{\normalfont\normalsize\bfseries}{}{0em}{}[.]
\titlespacing*{\section}{0em}{0.25\baselineskip}{0.5em}

%% LOGIC FOR CUSTOMIZING THE STATEMENT FOR INDIVIDUAL SCHOOLS
% set up common defines (commands and boolean flags)
% Sets up custom commands and flags for school-specific behaviors
% Also assigns default values, but these can be overwritten with school-specific
% values in the corresponding header file.
% Some default values are placeholders (marked in red) and these MUST be 
% overwritten in the school-specific headers!
\newcommand{\schoolname}{\textcolor{red}{COLLEGE\_NAME}\xspace} 
\newcommand{\schoolnamelong}{\textcolor{red}{COLLEGE\_NAME\_LONG}\xspace}
\newcommand{\schooladdress}{\textcolor{red}{47 Generic St., City, ST 99999}\xspace}
\newcommand{\schoolintrocourses}{\textcolor{red}{[school-specific intro course numbers]}\xspace}
\newcommand{\schooladvcourses}{\textcolor{red}{[school-specific AI/ML/NLP course numbers]}\xspace}
\newcommand{\chairname}{\textcolor{red}{Prof. Search Chair}\xspace}
\newcommand{\chairlastname}{\textcolor{red}{Prof. Chair}\xspace}
\newcommand{\chairtitle}{\textcolor{red}{Chair, Faculty Search Committee}\xspace}
\newcommand{\department}{Department of Computer Science\xspace}
\newcommand{\position}{\textcolor{red}{POSITION\_NAME (POSITION\_ID)}\xspace}

\newif\iflongdei
\longdeitrue 
\newif\ifappendix
\appendixtrue 
\newif\ifliberalarts
\liberalartstrue

\newcommand{\materials}{my latest CV, a graduate transcript, and statements on teaching, research, and commitment to diversity and inclusion.\xspace}

\newcommand{\coverteachingpara}{%
\textcolor{red}{[school-specific segment on teaching philosophy]}
}

\newcommand{\coverresearchpara}{%
\textcolor{red}{[school-specific segment on research]}
}

\newcommand{\lateachingintro}{%
Liberal Arts education played a formative role in my approach to teaching---my love of teaching was first sparked during my undergraduate studies at Harvey Mudd College.
That was where I got my first (small) taste of what it is like to be an instructor, through my participation in the CS department's ``grutoring'' program---similar to TA positions found at larger universities, but with more emphasis on direct interaction with students.
This early experience of engaging directly with individual students, and seeing the specific things they struggled with, strongly influenced how I think about teaching.
To this day, I approach teaching by starting from a student's perspective: for any given course, what is a student most likely to struggle with, and what's the most effective way to help them understand it?
While the specific answers to these questions will naturally vary with each course and each student cohort, there are nevertheless a few concrete strategies which, broadly speaking, I have found to be good \emph{starting points} for any course: \textbf{teaching with narratives}, \textbf{leveraging technology for interactivity}, and \textbf{building inclusive learning environments}.%
}

\newcommand{\genteachingintro}{%
\textcolor{red}{TODO WRITE ME}
}

\newcommand{\lanarrativeend}{%
This makes it particularly valuable for a Liberal Arts setting, which emphasizes drawing connections between all parts of a student's education.
In such a setting, one way I could imagine explicitly drawing out such connections in an intro-level course would be to build narratives based on content from other core courses.
For example, using real problems faced in Digital Humanities as motivating examples for lectures would help connect the lecture content to students' core humanities coursework.
This approach has the potential to hit ``two birds with one stone'', advancing the College's mission while simultaneously grounding the core CS concepts to make them more approachable.%
}

\newcommand{\gennarrativeend}{%
\textcolor{red}{TODO WRITE ME}
}

\newcommand{\lainteractionend}{I believe this solution is particularly well suited to the small-class environment of a liberal arts college.}

\newcommand{\geninteractionend}{%
\textcolor{red}{TODO WRITE ME}
}

% import the school-specific header to make the magic happen!
% Not an actual school; meant to provide placeholder values
% for school-specific variables for the purposes of compiling
% generic versions of the statements
\renewcommand{\schoolname}{\textcolor{red}{COLLEGE\_NAME}\xspace}
\renewcommand{\schoolintrocourses}{\textcolor{red}{[school-specific intro course numbers]}\xspace}
\renewcommand{\schooladvcourses}{\textcolor{red}{[school-specific AI/ML/NLP course numbers]}\xspace}
\longdeitrue
\appendixtrue
\liberalartstrue
%% END CUSTOMIZATION LOGIC

\title{SOME FANCY TITLE HERE}

\begin{document}
\maketitle

{\centering Teaching Statement --- Jonathan P. Chang \par}

\vspace{0.5\baselineskip}
% TODO different opening paragraph for non liberal arts colleges
Liberal arts education played a formative role in my approach to teaching---my love of teaching was first sparked during my undergraduate studies at Harvey Mudd College.
That was where I got my first (small) taste of what it is like to be an instructor, through my participation in the CS department's ``grutoring'' program---similar to TA positions found at larger universities, but with more emphasis on direct interaction with students.
I quickly fell in love with the rewarding experience of working with students to help them understand challenging concepts, which eventually led my interest in a teaching career.

Through my subsequent teaching experiences, I have developed a guiding philosophy for my teaching: I aim for learning environments where students can feel a sense of \emph{ownership} and \emph{accomplishment} over their learning.
Students in my classroom should not feel like mere receivers of knowledge, but instead should feel like active participants working together with me and each other to arrive at a shared understanding of the material.
During my PhD, I have had the chance to implement this philosophy in two disparate settings: a more traditional classroom setting as co-instructor of record for CS4300 ``Information and Language'',\footnote{\url{https://classes.cornell.edu/browse/roster/SP23/class/CS/4300}} and a more ad-hoc setting as a designer and instructor of data science workshops for the Cornell Center for Social Sciences (CCSS).\footnote{\url{https://socialsciences.cornell.edu/computing-and-data/workshops-and-training}}
These experiences serve to illustrate three concrete strategies that emerge from my teaching philosophy: \textbf{teaching with narratives}, \textbf{leveraging technology for interactivity}, and \textbf{building inclusive learning environments}.

\section{Teaching With Narratives}
%In the summer of 2022, I was designing a brand-new Introduction to Python workshop that I would teach for CCSS, meant for a target audience of social science students and researchers who may have never programmed before.
%My approach for making the code concepts of programming accessible to this audience was ground those abstract concepts in a social science narrative: using a real dataset sourced from a social science researcher at CCSS, the workshop asked students to imagine that they were using this data for their own research, then framed the lessons in terms of steps a researcher would need to take in preparing and analyzing the data.
%For example, the need to store and represent datapoints motivates the concept of variables, and the need to repeatedly perform common operations across all the data motivates the concepts of loops and functions.

%This offers an illustrative example of one of my preferred teaching strategies: organizing lesson plans around a central narrative, with each lesson having a clearly defined purpose within that narrative.
%This strategy arises directly from my core philosophy of having students feel ownership over their learning: I believe these narratives signal to students that the course is really designed \emph{for them}, in that they are gaining skills that directly connect to their own goals and needs as opposed to just rotely absorbing information with no rhyme or reason.
%Feedback from the workshop participants validates this belief, as several students told us that they learned much more from our workshop than in their previous attempts to learn Python or other programming skills.
%For instance, one student wrote in their feedback form ``I finally understood how to use Python; I have taken other classes at Cornell but none were so comprehensive as yours'', while another commented that ``This is one of the best tutorials I've been to. The instructors were knowledgeable, kind, and better at communicating/explaining code then most I've seen.''

One way I help students feel like a course is made \emph{for them} is to ground course content in \emph{narratives} that are relevant to their interests.
This is particularly helpful for reaching students whose backgrounds and interests are outside of Computer Science---for example, when I was developing an Introduction to Python workshop for CCSS, which was meant for a target audience of social science students who may have never programmed before, my approach was to construct a narrative around analyzing a real social science dataset from CCSS, grounding the core concepts of programming in steps a researcher would need to take in preparing and analyzing the data for social science research.
For example, the need to store and represent datapoints motivates the concept of variables, and the need to repeatedly perform common operations across all the data motivates the concepts of loops and functions.
Student feedback indicates that this approach was an effective and approachable way to communicate key concepts in programming:
\begin{quote}
    This is one of the best tutorials I've been to. The instructors were knowledgeable, kind, and better at communicating/explaining code then most I've seen.
\end{quote}

Looking ahead, I believe this strategy will be very applicable for teaching undergraduate intro-level CS courses, which similarly need to cater to students with little to no prior programming experience.
\narrativeendsent

\section{Leveraging Technology for Interactivity}
There has been growing interest in how new technologies for remote learning and collaboration could enable a transition away from the traditional paradigm of ``lectures in class, coursework outside of class''.
One notable approach, of course, has been to go all the way to the opposite extreme, the so-called ``flipped classroom'' model.
But I believe there is potential in the space between these two extremes: rather than viewing lectures and interactive coursework as separate entities that must always be kept apart, we should consider intermixing the two, creating a classroom environment that is neither lecture-style nor flipped-style but something altogether new.
During my time as co-instructor for CS4300, I took preliminary steps towards implementing these ideas in collaboration with Cornell's Active Learning Institute (ALI).

One step in this implementation was integrating live interactive coding into lectures.
Using Google Colaboratory allowed us to provide Python notebooks that students could experiment with during designated ``coding segments'' integrated within lectures.
These coding segments are are designed around a model of dialog between instructors and students: while parts of the code are written and demonstrated by the instructor, in other parts we turn things over to the students to come up with their own ideas.
Importantly, these student-led portions are not quizzes: there is often no one correct answer, and rather than being graded, they are used as conversation starters to have students discuss the different ways they approached the problem.
We additionally implemented non-coding interactive activities in a similar style, using tools such as polling and forms to facilitate discussion of open-ended questions.

Student feedback from a post-course survey conducted by ALI both show the promise of this approach and illuminate necessary next steps to develop it further.
Many students indicated that they found the interactive activities useful for improving their understanding of class concepts and staying more engaged.
Some example comments along these lines are ``[live coding] helped me learn and be more engaged because I was actively seeing concepts play out in real time.'' and ``I thought they [interactive activities] were great to discuss my thoughts and tinker with our learning with classmates.''
At the same time, other comments voiced criticisms of the format of the activities, especially when it comes to the live coding which could at times be unwieldy and hard to follow.
I am actively working on ways to address these shortcomings---for instance, looking into the use of Binder\footnote{\url{https://mybinder.org/}} as a more frictionless interactive coding environment---and look forward to integrating them into my teaching at \schoolname.

\section{Developing Course Content With Diverse Student Backgrounds in Mind}
As I've discussed, my core philosophy holds that students should feel like they are actively working together in their learning.
For this to work, every student must feel like they have a place and are able to contribute, regardless of their background.
To this end, I seek to develop my course content in ways that account for the fact that not every student has had the same opportunities in the past, with support systems to ensure that everyone can compete on the same level.

One major contribution I made in pursuit of this goal as co-instructor of CS4300 was my role in redeveloping the course project requirement.
Historically, the final project has been meant as an opportunity for students to showcase what they have learned by developing a novel information retrieval app.
% TODO check accuracy of this sentence with Cristian
While students have consistently found this to be one of the most exciting aspects of the class, one downside is that students who have had the privilege of prior app development experience---e.g., through internships or coding bootcamps---may find themselves with a head start compared to those who have not, as the latter group would need to first acquaint themselves with the basics of building an app before they can even get to thinking about the information retrieval aspect, which is where the actual intended educational value lies.
To address this, I oversaw the design and deployment of a new project server interface, which is designed to automatically handle the boilerplate aspects of app development/deployment (e.g., setting up a database connection) as well as a common starter template for all students to use, which automatically handles the work of laying out a user interface.

While the deployment of this new platform did not go perfectly smoothly---as with any new software, there ended up being a number of bugs that sometimes caused inconveniences and outages---student feedback neverthless suggests that it did help achieve my goal of empowering students to succeed and thrive on the project regardless of background experience.
One comment we received specifically called this out: ``I think this [project server] was really nice as it really leveled the playing field with regards to prior experience.''
Other comments even suggested that with this system in place, the final project was even able to serve as an accessible introduction to app development; for instance, `` I feel that it added to my knowledge about front-end and back-end programming.''

The project server goes hand in hand with other, less ambitious but still important, course design choices aimed at accounting for diverse student backgrounds and needs.
\iflongdei
I discuss these in more detail in my statement on diversity, equity, and inclusion.
\else
For instance, in recognition of the fact that not all students may be comfortable vocally contributing in class, I sought to develop alternative avenues of participation---such as the poll-based activities described previously.
\fi
Of course, I recognize that these constitute only small steps towards achieving a more equitable and inclusive classroom.
I look forward to continuing to search for ways to design a learning environment where anyone can thrive, and I am excited for the chance to learn from other faculty and students at \schoolname so I can incorporate and build upon the work they have done in this direction.

\section{Future Teaching Plans}
I have only just gotten started on my teaching journey.
As I look ahead to hopefully a new chapter at \schoolname, I am especially excited for the opportunity to teach some of the foundational courses, most notably \textcolor{red}{TODO school-specific course numbers will go here}.
While I have taught at a variety of levels throughout my prior career, foundational CS courses are what excite me the most---it is deeply satisfying for me to be able to successfully convey the core concepts of computer science, and the fact that these concepts are still novel to the students always makes for interesting conversations that may even lead me to see old ideas in a new light.
That said, I am still also passionate about teaching higher-level courses related to the research agenda I have laid out in my research statement, such as \textcolor{red}{TODO school-specific AI/ML/NLP course numbers}.
Finally, I also look forward to the opportunity to expand \schoolname's course offerings; in particular, I would be interested in developing a new course on computational social science---lying at the intersection of STEM and the humanities, I believe such a course would be a perfect fit for a computer science department at a liberal arts institution.

\ifappendix
\vspace{\baselineskip}
\section{Appendix: Selected Comments From Student Evaluations}
Below are some selected comments from student evaluations and surveys which did not fit in any particular section above, but which I felt are otherwise indicative of my teaching philosophy and its outcomes.
Comments are grouped by source.
The source data---that is, the evaluations themselves in the raw format they were provided to me---is available in the supplementary materials attached to my submission.

%\noindent\underline{CS4300 ALI survey}

\noindent\underline{CS4300 midterm evaluations}
\begin{itemize}
    \item I really enjoy Jonathan's teaching, he's very passionate when he speaks which keeps me engaged.
    \item Great at engaging with the audience and working to keep attention
\end{itemize}

\noindent\underline{CS4300 end-of-semester evaluations}
\begin{itemize}
    \item Made concepts much easier to understand and was engaging during the lectures, and was super helpful during his office hours
    \item Very engaging and well-spoken. Learned the most from Professor Chang's lectures. I thought the use of the active note taking assisted in keeping me engaged and taking useful notes.
    \item Genuinely. Prof. Chang was great at teaching, so his lectures were pretty helpful
\end{itemize}

\noindent\underline{CCSS workshops}\\
\emph{[Note: comments are taken from a survey question about my Introduction to Python workshop; references to ``this'' and ``it'' are referring to that workshop.]}
\begin{itemize}
    \item Thank you...I finally understood how to use Python. I have taken other classes at Cornell before but none were so comprehensive as yours.
    \item It was a great introduction, friendly, and overall positive environment. It also was taught extremely well, at the perfect pace and everything was explained very clearly. It also gave students the opportunity to get to know instructors and each other. The basics were taught very well.
\end{itemize}

\else
%
\fi

\end{document}
