\documentclass[12pt,letterpaper]{article}
\usepackage[margin=0.75in]{geometry}
\usepackage{graphicx}
%\usepackage[bf,tiny,compact]{titlesec}
\usepackage{times}
\usepackage{xcolor}
\usepackage{xspace}
\usepackage{enumerate}
\usepackage{enumitem}
\usepackage{titlesec}
\usepackage{etoolbox}
\usepackage{hyperref}


\setlist{nosep}

\makeatletter
\def\@maketitle{%
    {\centering\fontsize{12pt}{14pt}\selectfont\textbf{\@title}\par}
}
\patchcmd{\@footnotetext}{\footnotesize}{\fontsize{10pt}{12pt}\selectfont}{}{}
\makeatother

\titleformat{\section}[runin]{\normalfont\normalsize\bfseries}{}{0em}{}[.]
\titlespacing*{\section}{0em}{0.25\baselineskip}{0.5em}

%% LOGIC FOR CUSTOMIZING THE STATEMENT FOR INDIVIDUAL SCHOOLS
% set up common defines (commands and boolean flags)
% Sets up custom commands and flags for school-specific behaviors
% Also assigns default values, but these can be overwritten with school-specific
% values in the corresponding header file.
% Some default values are placeholders (marked in red) and these MUST be 
% overwritten in the school-specific headers!
\newcommand{\schoolname}{\textcolor{red}{COLLEGE\_NAME}\xspace} 
\newcommand{\schoolnamelong}{\textcolor{red}{COLLEGE\_NAME\_LONG}\xspace}
\newcommand{\schooladdress}{\textcolor{red}{47 Generic St., City, ST 99999}\xspace}
\newcommand{\schoolintrocourses}{\textcolor{red}{[school-specific intro course numbers]}\xspace}
\newcommand{\schooladvcourses}{\textcolor{red}{[school-specific AI/ML/NLP course numbers]}\xspace}
\newcommand{\chairname}{\textcolor{red}{Prof. Search Chair}\xspace}
\newcommand{\chairlastname}{\textcolor{red}{Prof. Chair}\xspace}
\newcommand{\chairtitle}{\textcolor{red}{Chair, Faculty Search Committee}\xspace}
\newcommand{\department}{Department of Computer Science\xspace}
\newcommand{\position}{\textcolor{red}{POSITION\_NAME (POSITION\_ID)}\xspace}

\newif\iflongdei
\longdeitrue 
\newif\ifappendix
\appendixtrue 
\newif\ifliberalarts
\liberalartstrue

\newcommand{\materials}{my latest CV, a graduate transcript, and statements on teaching, research, and commitment to diversity and inclusion.\xspace}

\newcommand{\coverteachingpara}{%
\textcolor{red}{[school-specific segment on teaching philosophy]}
}

\newcommand{\coverresearchpara}{%
\textcolor{red}{[school-specific segment on research]}
}

\newcommand{\lateachingintro}{%
Liberal Arts education played a formative role in my approach to teaching---my love of teaching was first sparked during my undergraduate studies at Harvey Mudd College.
That was where I got my first (small) taste of what it is like to be an instructor, through my participation in the CS department's ``grutoring'' program---similar to TA positions found at larger universities, but with more emphasis on direct interaction with students.
This early experience of engaging directly with individual students, and seeing the specific things they struggled with, strongly influenced how I think about teaching.
To this day, I approach teaching by starting from a student's perspective: for any given course, what is a student most likely to struggle with, and what's the most effective way to help them understand it?
While the specific answers to these questions will naturally vary with each course and each student cohort, there are nevertheless a few concrete strategies which, broadly speaking, I have found to be good \emph{starting points} for any course: \textbf{teaching with narratives}, \textbf{leveraging technology for interactivity}, and \textbf{building inclusive learning environments}.%
}

\newcommand{\genteachingintro}{%
\textcolor{red}{TODO WRITE ME}
}

\newcommand{\lanarrativeend}{%
This makes it particularly valuable for a Liberal Arts setting, which emphasizes drawing connections between all parts of a student's education.
In such a setting, one way I could imagine explicitly drawing out such connections in an intro-level course would be to build narratives based on content from other core courses.
For example, using real problems faced in Digital Humanities as motivating examples for lectures would help connect the lecture content to students' core humanities coursework.
This approach has the potential to hit ``two birds with one stone'', advancing the College's mission while simultaneously grounding the core CS concepts to make them more approachable.%
}

\newcommand{\gennarrativeend}{%
\textcolor{red}{TODO WRITE ME}
}

\newcommand{\lainteractionend}{I believe this solution is particularly well suited to the small-class environment of a liberal arts college.}

\newcommand{\geninteractionend}{%
\textcolor{red}{TODO WRITE ME}
}

% import the school-specific header to make the magic happen!
% Not an actual school; meant to provide placeholder values
% for school-specific variables for the purposes of compiling
% generic versions of the statements
\renewcommand{\schoolname}{\textcolor{red}{COLLEGE\_NAME}\xspace}
\renewcommand{\schoolintrocourses}{\textcolor{red}{[school-specific intro course numbers]}\xspace}
\renewcommand{\schooladvcourses}{\textcolor{red}{[school-specific AI/ML/NLP course numbers]}\xspace}
\longdeitrue
\appendixtrue
\liberalartstrue
% link paragraph commands to the appropriate concrete implementation
\ifliberalarts
\newcommand\intropar\lateachingintro
\newcommand\narrativeendsent\lanarrativeend
\newcommand\interactionimprove\lainteractionend
\else
\newcommand\intropar\genteachingintro
\newcommand\narrativeendsent\gennarrativeend
\newcommand\interactionimprove\geninteractionend
\fi
%% END CUSTOMIZATION LOGIC

\title{Teaching Statement}

\begin{document}
\maketitle

{\centering Jonathan P. Chang \par}

\vspace{0.5\baselineskip}
\intropar
%I quickly fell in love with the rewarding experience of working with students to help them understand challenging concepts, which eventually led my interest in a teaching career.
%
%Through my subsequent teaching experiences, I have developed a guiding philosophy for my teaching: I aim for learning environments where students can feel a sense of \emph{ownership} and \emph{accomplishment} over their learning.
%Students in my classroom should not feel like mere receivers of knowledge, but instead should feel like active participants working together with me and each other to arrive at a shared understanding of the material.
%During my PhD, I have had the chance to implement this philosophy in two disparate settings: a more traditional classroom setting as co-instructor of record for CS4300 ``Information and Language'',\footnote{\url{https://classes.cornell.edu/browse/roster/SP23/class/CS/4300}} and a more ad-hoc setting as a designer and instructor of data science workshops for the Cornell Center for Social Sciences (CCSS).\footnote{\url{https://socialsciences.cornell.edu/computing-and-data/workshops-and-training}}
%These experiences serve to illustrate three concrete strategies that emerge from my teaching philosophy: \textbf{teaching with narratives}, \textbf{leveraging technology for interactivity}, and \textbf{building inclusive learning environments}.


\section{Teaching With Narratives}
One fundamental challenge in CS education that I became aware of early on, which is particularly relevant in intro CS settings, is that key CS concepts can often feel abstract and ungrounded to a newcomer.
I have found through experience that one effective way to address this challenge is by grounding those abstract concepts in \emph{narratives} that students can relate to.

One of the clearest examples of this strategy is an Introduction to Python workshop that I developed for the Cornell Center for Social Sciences (CCSS)\footnote{\url{https://socialsciences.cornell.edu/computing-and-data/workshops-and-training}} as one of their Data Science Fellows.
This workshop was meant for a target audience of social science students who may have never programmed before.
I therefore centered the workshop on a narrative about analyzing a real social science dataset from CCSS, grounding core programming concepts in concrete steps a researcher would need to take in preparing and analyzing the data for social science research.
For example, the concept of variables is motivated by the need to store and represent datapoints, while loops and functions are motivated by the need to repeatedly perform common operations across all the data.
Student feedback indicates that this was an effective and approachable way to communicate key concepts in programming, as best illustrated by this comment from a student who had previously found little success trying to learn Python until they came to this workshop:\footnote{``Sam'' in the quote refers to my fantastic co-instructor for the workshop, Samantha De Leon Sautu!}
\begin{quote}
    Thank you, Sam and Jonathan - I finally understood how to use Python.
\end{quote}

Of course, what counts as a relatable narrative will depend greatly on context.
For instance, a upper-level CS course will require a very different kind of narrative compared to intro-level courses like my Python workshop.
As an example of how I approach this strategy for more advanced courses, during my role as co-instructor of record for CS4300 ``Information and Language'',\footnote{\url{https://classes.cornell.edu/browse/roster/SP23/class/CS/4300}} I grounded some key machine learning concepts by drawing connections to recent advances in language technology that students may have heard about in the news.
Students seemed to recognize my efforts: on the course evaluation question ``Did the lecturer motivate the course content
and place it in the context of your major or your overall engineering
education?'' students rated me an average 4.12 out of 5.

These examples show the flexibility of this strategy for helping students at all levels connect the material in their coursework to the bigger picture.
\narrativeendsent

\section{Leveraging Technology for Interactivity}
Hands-on experience is another effective method to help students wrap their heads around material that might be challenging or confusing.
I believe that recent advances in collaboration technology, despite being often associated with remote learning and flipped classrooms, are powerful tools for providing such hands-on experience in traditional classroom environments as well.
For example, with input from Cornell's Active Learning Institute (ALI), my teaching in CS4300 made heavy use of Google Colaboratory as a shared interactive coding environment.
This allowed students to engage with lecture material in real time, both by following along with code I demonstrated in front of the room, and also by partaking in open-ended coding activities where they were given the chance to try out various ideas and then share what they came up with in a class discussion.
I also leveraged online forms and polling alongside old-fashioned pen-and-paper handouts to implement similar collaborative activites in non-coding parts of the lectures as well.
Comments from an ALI-administered survey show how students perceived the benefits of these activities:

\begin{quote}
    [Live coding] helped me learn and be more engaged because I was actively seeing concepts play out in real time.
\end{quote}
\begin{quote}
    I thought they [interactive activities] were great to discuss my thoughts and tinker with our learning with classmates.
\end{quote}

At the same time, other comments point to shortcomings in this approach that I have concrete plans to address in my future teaching.
In particular, some students remarked that live coding could get hard to follow, since falling behind early on would leave them out-of-sync throughout the rest of the live coding session.
To address this, I plan to adapt techniques from my CCSS workshop teaching, such as having designated ``checkpoints'' throughout live coding sessions to give students a chance to catch up, during which an instructor
\ifliberalarts
%
\else
or TA
\fi
could reach out to individual students who may be particularly stuck on some part.
\interactionimprove

\section{Building Inclusive Learning Environments}
Part of thinking about course design from a student perspective involves acknowledging that not every student has had the same opportunities in the past, and thus I as an instructor must actively work to help all students participate on the same level.
This has design implications at all levels of a course---that is, both inside and outside the classroom.

Inside the classroom, one of my recent focuses has been improving inclusivity in in-class participation.
Differences in background or circumstance can often mean that some students are less comfortable than others with vocally asking or answering questions in front of the whole class, or unable to do so at all.
To ensure such students still feel included, I leveraged my interest in collaboration technology to design alternative avenues for participation in CS4300.
These include the aforementioned polls and quizzes, as well as online discussions via platforms like Canvas and Ed\footnote{\url{https://edstem.org/}, a platform similar to Piazza} with the option of participating anonymously.

Outside the classroom, accounting for differences in background can improve fairness in the design of course assignments.
For example, the CS4300 final project has historically involved creating an app that incorporates information retrieval techniques from the class, but consequently students who have had prior opportunities to practice app development skills (e.g., through an internship) may find themselves having a smoother experience than students who haven't had access to such opportunities.
To address this, I oversaw the development of a brand-new project server and template designed to abstract away the boilerplate app development details, allowing students to focus on the part we ultimately want to evaluate on: the information retrieval.
Student feedback from course evaluations provide evidence that this indeed made for a more quitable environment:
\begin{quote}
I think this [project server] was really nice as it really leveled the playing field with regards to prior experience.
\end{quote}

These efforts appear to have made a noticeable difference in the classroom, with students 4.44 out of 5 on diversity and inclusion in course evaluations, where 5 corresponds to ``Actively inclusive''.
That said, I also recognize that there is always more I can do to further improve in this direction, and I look forward to working with the diverse and talented faculty at \schoolname to share and develop techniques for designing a classroom that is welcoming to all.
\iflongdei
I discuss these ideas and my future plans for building inclusive environments in more detail in the statement on diversity, equity, and inclusion that was submitted as part of my application.
\else
%
\fi

\section{Future Teaching Plans}
From the beginnings of my teaching as a Harvey Mudd ``grutor'' to my classroom and workshop teaching at Cornell, I have have had the chance to experience teaching at levels ranging from introductory programming to advanced AI and machine learning topics.
Throughout these experiences, I have come to find that foundational CS courses are what excite me the most---it is deeply satisfying for me to be able to successfully convey the core concepts of computer science, and the fact that these concepts are still novel to the students always makes for interesting conversations that may even lead me to see old ideas in a new light.
As such, looking ahead to a future career at \schoolname, I would be especially excited for the chance to teach foundational courses like \schoolintrocourses.
That said, I would also be interested in teaching higher-level courses related to the research agenda I have laid out in my research statement, such as \schooladvcourses.
Finally, I also look forward to the opportunity to expand \schoolname's course offerings; in particular, I would be interested in developing a new course on computational social science---lying at the intersection of STEM and the humanities, I believe such a course would be a perfect fit for a computer science department at a liberal arts institution.


\ifappendix
\vspace{\baselineskip}
\section{Appendix: Selected Comments From Student Evaluations}
%Below are some selected comments from student evaluations and surveys which did not fit in any particular section above, but which I felt are otherwise indicative of my teaching philosophy and its outcomes.
%Comments are grouped by source.
Note: full comments are available in the supplementary materials attached to my submission.

%\noindent\underline{CS4300 ALI survey}

\noindent\underline{CS4300 midterm evaluations}
\begin{itemize}
    \item I really enjoy Jonathan's teaching, he's very passionate when he speaks which keeps me engaged.
    \item Great at engaging with the audience and working to keep attention.
\end{itemize}

\noindent\underline{CS4300 end-of-semester evaluations}
\begin{itemize}
    \item Made concepts much easier to understand and was engaging during the lectures, and was super helpful during his office hours.
    \item Very engaging and well-spoken. Learned the most from Professor Chang's lectures. I thought the use of the active note taking assisted in keeping me engaged and taking useful notes.
    \item Genuinely. Prof. Chang was great at teaching, so his lectures were pretty helpful.
\end{itemize}

\noindent\underline{CCSS workshops}\\
\emph{[Note: comments are taken from a survey question about my Introduction to Python workshop; references to ``this'' and ``it'' are referring to that workshop.]}
\begin{itemize}
    \item I've been a grad student for several years and decided to learn python to gain another skill. (My field uses Stata and R). This is one of the best tutorials I've been to. The instructors were knowledgeable, kind, and better at communicating/explaining code then most I've seen. 
    \item It was a great introduction, friendly, and overall positive environment. It also was taught extremely well, at the perfect pace and everything was explained very clearly. It also gave students the opportunity to get to know instructors and each other. The basics were taught very well.
\end{itemize}

\else
%
\fi

\end{document}
