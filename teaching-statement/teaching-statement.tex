\documentclass[11pt,letterpaper]{article}
\usepackage[margin=0.75in]{geometry}
\usepackage{graphicx}
%\usepackage[bf,tiny,compact]{titlesec}
\usepackage{times}
\usepackage{cite}
\usepackage{natbib}
\usepackage{titlesec}
\usepackage{etoolbox}
\usepackage{hyperref}


\makeatletter
\def\@maketitle{%
    {\centering\fontsize{12pt}{14pt}\selectfont\textbf{\@title}\par}
}
\patchcmd{\@footnotetext}{\footnotesize}{\fontsize{10pt}{12pt}\selectfont}{}{}
\makeatother


\renewcommand{\bibsection}{\noindent\textbf{References (works to which I contrubted have my name \underline{underlined})}\vspace{-6pt}}
\setlength{\bibsep}{0pt} % or use whatever dimension you want
\renewcommand{\bibfont}{\fontsize{10pt}{12pt}\selectfont} % or any other  appropriate font command


\renewcommand{\section}[1]{\vspace{0.25\baselineskip}\noindent\textbf{#1.}}
\renewcommand{\subsection}[1]{\vspace{0.25\baselineskip}\noindent\textit{#1.}}

\title{SOME FANCY TITLE HERE}

\begin{document}
\maketitle

\begin{center}
Teaching Statement --- Jonathan P. Chang
\end{center}

% TODO different opening paragraph for non liberal arts colleges
Liberal arts education played a formative role in my approach to teaching---my love of teaching was first sparked at Harvey Mudd College, where I did my undergraduate degree.
Specifically, Harvey Mudd was where I got my first small taste of what it is like to be an instructor, through my participation in the CS department's ``grutoring'' program---similar to the TA positions found at larger universities, but with a greater emphasis on direct interaction with students.
Though I initially joined the program just out of curiosity (and admittedly also for the chance to get paid!), I quickly fell in love with both the challenge of finding the best way to explain a difficult concept to a student, and the satisfaction of seeing the ``aha'' moment where they finally get it.
Since then, my later experiences as an actual instructor have been spent chasing those ``aha'' moments---and more concretely, developing and implementing a strategy to make those moments happen more often and reliably.

Subsequently, I have developed the following guiding philosophy for my teaching: I aim to design learning environments where students can feel a sense of \emph{ownership} and \emph{accomplishment} over their learning.
Students in my classroom should not feel like mere receivers of knowledge, but instead should feel like active participants working together with me and with each other to arrive at a shared understanding of the material.
During my PhD, I have had the chance to implement this philosophy in two disparate settings: a more traditional classroom setting as co-instructor of CS4300 ``Information and Language'', and a more personal and ad-hoc setting as a designer and instructor of data science workshops for the Cornell Center for Social Sciences.
These experiences serve to illustrate three concrete strategies that emerge from my teaching philosophy: \textbf{narrative-driven lesson plans}, \textbf{technological innovations for interactivity}, and \textbf{developing curricula with diverse student backgrounds in mind}.

\end{document}
