\renewcommand{\schoolname}{Harvey Mudd\xspace}
\renewcommand{\schoolnamelong}{Harvey Mudd College\xspace}
\renewcommand{\schoolintrocourses}{CSCI 005, CSCI 042, CSCI 060, and CSCI 070\xspace}
\renewcommand{\schooladvcourses}{CSCI 151: Artificial Intelligence, CSCI 158: Machine Learning, and CSCI 159: Natural Language Processing\xspace}
\renewcommand{\chairname}{Prof. Zach Dodds\xspace}
\renewcommand{\chairlastname}{Prof. Dodds\xspace}
\renewcommand{\chairtitle}{Co-chair, Faculty Search Committee\xspace}
\renewcommand{\schooladdress}{301 Platt Blvd, Claremont, CA 91711\xspace}
\renewcommand{\position}{Assistant Professor of Computer Science (AcademicJobsOnline ID \#24986)\xspace}
\longdeitrue
\appendixtrue
\liberalartstrue

\renewcommand{\materials}{my latest CV and statements on teaching, research, and diversity.\xspace}

\renewcommand{\lateachingintro}{%
Harvey Mudd played a formative role in my approach to teaching: my love of teaching was first sparked during my experience as an undergraduate there.
Specifically, my first (small) taste of what it is like to be an instructor came from my participation in the CS department's grutoring program.
As a grutor for seven semesters during my time at Mudd,\footnote{Starting with CS 60 in my second semester, then three semesters of CS 70, and one semester each in CS 131, 140, and 151.} I had the chance to engage directly with individual students and see the specific things they struggled with, an experience which strongly influenced how I think about teaching.
To this day, I approach teaching by starting from a student's perspective: for any given course, what is a student most likely to struggle with, and what's the most effective way to help them understand it?
While the specific answers to these questions will vary with each course and each student cohort, there are nevertheless a few concrete strategies which I have found to be good \emph{starting points} for any course: \textbf{teaching with narratives}, \textbf{leveraging technology for interactivity}, and \textbf{building inclusive learning environments}.%
}

\renewcommand{\lateachingend}{%
Finally, I also look forward to the opportunity to expand \schoolname's course offerings; in particular, I would be interested in developing a new course on Computational Social Science, a topic that fits well with the interdisciplinary nature of a Liberal Arts education and especially Harvey Mudd's focus on training well-rounded students.
}

\renewcommand{\laresearchintro}{%
This research agenda is a particularly good fit for Harvey Mudd, advancing the College's mission by giving undergraduate students an opportunity to work with cutting-edge technology and simultaneously study the impact of this technology on society.
}

\renewcommand{\laresearchclosing}{%
Overall, I believe that my research agenda has a lot to offer for Harvey Mudd: its cutting-edge nature will help the College expand its research footprint, its combination of research and engineering components opens up opportunities for undergraduates of varying backgrounds and interests, and its focus on real-world impact will help to advance Harvey Mudd's mission.
}

\renewcommand{\deioutreach}{%
To this end, I would be interested in getting involved once more in the Peer Academic Liaison program or other similar efforts. I also see potential in pioneering a new department-level program that could offer more targeted support and advice for CS students.%
}
